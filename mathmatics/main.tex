\documentclass[uplatex]{jsarticle}
\usepackage{listings, plistings, theorem, bm, amsmath, amssymb}
\lstset{
  basicstyle={\ttfamily},
  identifierstyle={\small},
  commentstyle={\smallitshape},
  keywordstyle={\small\bfseries},
  ndkeywordstyle={\small},
  stringstyle={\small\ttfamily},
  frame={tb},
  breaklines=true,
  columns=[l]{fullflexible},
  numbers=left,
  xrightmargin=0zw,
  xleftmargin=3zw,
  numberstyle={\scriptsize},
  stepnumber=1,
  numbersep=1zw,
  lineskip=-0.5ex,
}
\renewcommand{\lstlistingname}{ソースコード}

\theorembodyfont{\normalfont}
\newtheorem{definition}{定義}
\newtheorem{theorem}{定理}

\title{情報理工学系研究科 院試数学}
\author{mqcmd196}

\begin{document}
\section{線形代数}

\subsection{諸知識}

\subsection{ベクトル空間,写像,同型}
\begin{definition}
    \label{def:1次独立と1次従属}
    1次独立と1次従属\\
    ベクトルの組$\bm{a_1}, \bm{a_2}, ..., \bm{a_m}$について,1次関係$$c_1\bm{a_1} + c_2\bm{a_2}, ..., c_m\bm{a_m} = \bm{0} \, (c_i \in \mathbb{R})$$が成り立つのは,自明な1次関係つまり,$c_1 = c_2 = ... = c_m = 0$の場合のみのとき,$\bm{a_1}, \bm{a_2}, ..., \bm{a_m}$は1次独立であるという.また1次独立でないとき,すなわち$$c_1\bm{a_1} + c_2\bm{a_2}, ..., c_m\bm{a_m} = \bm{0}$$をみたす$c_1, c_2, ..., c_m$で,そのうちの少なくとも1つが0でないものが存在するとき,1次従属であるという.
\end{definition}

\begin{definition}
    部分空間\\
    ベクトル空間$V$の空でない部分集合$W$が,$V$における和とスカラー倍の演算によってベクトル空間になるとき,$W$を$V$の部分空間という.
\end{definition}

\begin{theorem}
    ベクトル空間$V$の部分集合$W$が部分空間であるための必要十分条件
    \begin{enumerate}
        \item $W \neq \phi$
        \item 任意の$\bm{a}, \bm{b} \in W$と任意の$\lambda, \mu \in \mathbb{R}$に対して$\lambda\bm{a} + \mu\bm{b} \in W$
    \end{enumerate}
\end{theorem}

\begin{definition}
    像空間と核空間\\
    $V, W$を2つのベクトル空間,$f:V \to W$を線形写像とするとき,$${\rm Im} f = \{f(\bm{x}) | \bm{x} \in V\}$$を$f$の像空間,$${\rm Ker} f = \{\bm{x} \in V | f(\bm{x}) = \bm{0}\}$$を$f$の核空間という.
\end{definition}

\end{document}